\documentclass{beamer}

\usepackage[ngerman]{babel}

\usetheme{Warsaw}
\usecolortheme{beaver}

\usepackage[utf8]{inputenc}
\usepackage[T1]{fontenc}

\usepackage{hyphenat}
\hyphenation{Mathe-matik wieder-gewinnen}

\usepackage{biblatex}
\addbibresource{literaturebeamer.bib}

\usepackage{hyperref}
\hypersetup{
	colorlinks=true,
	linkcolor=blue
}

\AtBeginSection[]
{
	\begin{frame}
		\frametitle{Inhaltsverzeichnis}
		\tableofcontents[currentsection]
	\end{frame}
}


\newcommand{\rund}[1]{\left(#1\right)}
\newcommand{\eck}[1]{\left[#1\right]}
\newcommand{\intd}[1]{\text{d}#1}
\title{Meine erste \LaTeX Präsentation}
\author{Mein Name}
\date{\today}
\begin{document}
	
\frame{\titlepage}

\begin{frame}
	\frametitle{Inhaltsverzeichnis}
	\tableofcontents
\end{frame}

\section{Mein erstes Kapitel}
\begin{frame}
	\frametitle{Meine erste richtige Folie}
	Einfach Text reinschreiben, der wird dann angezeigt. \\
	$4 + 5 = 9$
	\begin{equation}
		4 + 5 = 9
	\end{equation}
\end{frame}
\section{Mein zweites Kapitel}
\begin{frame}
	\frametitle{Meine erste richtige Folie}
	Einfach Text reinschreiben, der wird dann angezeigt. \\
	$4 + 5 = 9$
	\begin{equation}
		4 + 5 = 9
	\end{equation}
\end{frame}

\begin{frame}
	\frametitle{Eine Folie mit Stichpunkten}
	\begin{itemize}
		\item Ein Stichpunkt.
		\begin{itemize}
			\item 
		\end{itemize}
		\item Noch ein Stichpunkt.
	\end{itemize}
\end{frame}

\end{document}